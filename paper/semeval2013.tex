\documentclass[11pt,letterpaper]{article}
\usepackage{naaclhlt2013}
\usepackage{times}
\usepackage{latexsym}
\setlength\titlebox{6.5cm}    % Expanding the titlebox
\usepackage{float}
\floatstyle{boxed}
\restylefloat{figure}

%% already told Els that this was the title, can't change it now.
\title{HLTDI: CL-WSD Using Markov Random Fields for SemEval-2013 Task 10}

\author{Alex Rudnick, Can Liu and Michael Gasser\\
	    Indiana University, School of Informatics and Computing \\
	    {\tt \{alexr,liucan,gasser\}@indiana.edu}}

\date{}

\begin{document}
\maketitle

%what resource did we use, 
\begin{abstract}


We present our entries for the SemEval-2013 cross-language word-sense
disambiguation task \cite{task10}. We submitted three systems based
on classifiers trained on local context features, with some elaborations.
Our three systems, in increasing order of complexity, were: maximum entropy
classifiers trained to predict the desired target-language phrase using only monolingual features (we called this system ``L1"); similar classifiers, but with the desired target-language
phrase for the other four languages as features (``L2"); and lastly, networks
of five classifiers, over which we do loopy belief propagation in an attempt to
solve the classification task jointly (``MRF").
\end{abstract}

\section{Introduction}
In the cross-language word-sense disambiguation (CL-WSD) task, given an
instance of an ambiguous word used in a context, we want to predict the
appropriate translation into some target language. This setting for WSD has an
immediate application in machine translation, since many words have many
possible translations.

Framing lexical ambiguities in this way, as an explicit classification task,
has been shown to be improve machine translation even in the case of
phrase-based SMT systems (cite Carpuat and Wu), which can mitigate the
ambiguities through the use of a language model and phrase-tables with
multi-word phrases.
CL-WSD has been shown useful for statistical machine translation (cite Carpuat and Wu), although in future work we are particularly interested in applying it to rule-based systems. (XXX: is this relevant?)

In the Semeval-2013 CL-WSD task \cite{task10}, we are asked to build a system that can provide
translations for twenty ambiguous English nouns in their contexts. The five target languages in the shared task are Spanish, Dutch, German, Italian and French. There were two settings for the evaluation, ``best" and ``oof". In either case, systems may present multiple possible answers for a given translation, although in the ``best" setting, the first answer is given more weight, which encourages only returning the one-best. In the ``oof" setting, systems are encouraged to return the top-five most likely translations. For a complete explanation of the settings, please see the shared task description \cite{task10}.

%% consider: maybe move this to related work?
Following the work of Lefever and Hoste
\shortcite{lefever-hoste-decock:2011:ACL-HLT2011}, we wanted to develop systems
that make use of multiple bitext corpora for the CL-WSD task.
ParaSense, the system of Lefever and Hoste, takes into account evidence from all of the available parallel corpora. Let $S$ be the set of five target languages and $t$ be the particular target language of interest at the moment; ParaSense creates bag-of-words features from the translations of the target sentence into the languages $S - \lbrace{t \rbrace}$. Given corpora that are parallel over many languages, this is straightforward to do at training time, however at testing time it requires the use of a complete MT system into the four other languages, which is computationally prohibitive. Thus in our work, we have developed systems that make use of many parallel corpora but require neither a locally running MT system nor access to an online translation API.

We presented three systems in this competition, which were variations on the
theme of a maximum entropy classifier for each ambiguous noun, trained on local
context features similar to those used in previous work and familiar from the
WSD literature.

Our systems had similar results, but at the time of the evaluation, our simplest system came in first place for the out-of-five evaluation for three languages (Spanish, German, and Italian).
However, after the evaluation, we fixed a simple (slightly embarrassing) bug in our MRF code, which resulted in the MRF system posting even better results for the OOF evaluation.

on the \emph{oof} evaluation, we had the best results for Spanish, German, and Italian.
All of our systems beat the ``most-frequent sense" baseline in every case.


Our three systems made use of the same training data, which we extracted from
the Europarl Intersection corpus, meaning that the English-language source
sentences were identical for each of the five corpora. But the source-language
sentences need not be identical, and we plan to explore relaxing this
restriction in future work.

In the following sections, we will describe our three systems, followed by the
common preprocessing steps that we performed, and then discuss results and
future work.

We will describe our systems in a increasing order of complexity.
\section{L1}
preprocessing, training, testing\\

The one-layer classifier (which we call L1), is a Maximum Entropy (Multinomial Logistic Regression) classifier that uses only monolingual features from English.
Although this shared task is performed in an unsupervised manner (there is no annotated corpora), L1 was constructed in a supervised way.
With bitext of English and target language available from EuroParl, we extracted training instances from aligned sentences, where we used rich local context features(described below).
The L1 was trained for each target language and for each word of interest. 

An important part of building L1 is extracting relevant and good training instances. 
(We feel that it would be fairer if a standard way is provided..but maybe not it's unsupervised anyway XXX ???)
We used alignments from Berkeley Aligner to locate the corresponding translation for a source English word,
and lemmatize this translation using TreeTagger to produce a label.  
The quality of training data relies much on the alignments and lemmatization, in Els work XXX, manual aligning and lemmatizing was involved to provide an upper bound.
In it an  


Once we find translation pairs that we want to include in the training data
(mappings from English-language sentences and instances of the ambiguous word
in question to the appropriate target-language lemma), we extract features from
the English-language sentence. 

Several steps of preprocessing were needed. We first POS tagged the sentences, since we are only interested in nouns.
Then align the words in each sentence pair, and lemmatize the target sentence.
After locating words of interest in the
Europarl Intersection corpus, training instances were extracted, and a maxent
classifier was trained over local context features similar to those used by Lefever and
Hoste.

\begin{figure}
  \begin{itemize}
  \item word form
  \item word tag (word with tag?)
  \item word lemma
  \item for words in the three-word window on both sides
  \item wordform, lemma, pos tag
  \item for words in a five-word window on both sides
  \item bigrams and tagged bigrams (just in case)
  \end{itemize}
  \label{features}
  \caption{some features}
\end{figure}

Note: word tag is different from word with tag (so as for bigram and bigram
with tag). eg. ``NN" is a tag for ``bank", but ``bank/NN" is the word with tag.
We include this because tags are good indicator of word senses, and the senses
of context helps disambiguate the target word. 

\section{L2}
As in the work of Lefever and Hoste, the ``L2" model and MRF model both use
translation for four other languages as features, but in different ways.  ``L2"
model is a maxent classifier with features 1) from English source sentence (the
same features as L1), 2) translation for the target word into other four
languages.  At training time, these translations are extracted from the aligned
sentences in Europarl Intersection data.  At testing time, the translations are
estimated using the L1 classifiers. 

%Our "L2" model builds on the L1 model, but with the translations of the word of
%interest into the four other target languages as additional features.
The translations seems to be an important factor in the performance of L2,
Lefever and Hoste used a MT system for the translations instead.
???

%Unfortunately, it is difficult to evaluate which one gives better performance.
%It would be a fairer comparison of the WSD systems if a standardized way %of
%translations So as in the work of Lefever and Hoste, our L2 classifiers make
%use of several aligned bitext corpora, but without relying on a complete MT
%system.


\section{MRF}
Our "MRF" model builds a Markov network of L1 classifiers in an effort to find
the best translation into all five target languages jointly. This network has
nodes that correspond to the distribution produced by the L1 classifiers, and
edges with pairwise potentials that are derived from the joint probabilities of
target-language labels occurring together in the training data. We optimize
assignments over the five output variables using loopy belief propagation

The idea of MRF is that translations for five target languages must agree with
each other, since these languages are all related.  It is then natural to ask:
how related are they and can MRF represent different closeness between
languages?  ???  We know that Dutch is more related to German than with
Spanish, Italian and French. Can this closeness be represented by the pairwise
potentials?

There was some concern about pairwise potential in MRF, which is joint probability. Consider a word which occurs 500 times in the training data, it could co-occur with
We had some concern about pairwise potential in MRF, which is joint
probability. Consider a word which occurs 500 times in the training data, it
could co-occur with

The future work should experiment on different weight we put on the pairwise
potentials...

problem with joint probability as potentials, conditional probability as
potentials, we could abuse the mathematics of MRF a little bit and experiment
on this approximate potential.

\section{Resources and tools}
The Europarl Intersection corpus.

For simplicity and comparability with previous work, we worked entirely with
the Europarl Intersection corpus provided by the competition organizers.
	

\section{preprocessing steps}
tools: NLTK, Stanford Tagger, Berkeley Aligner, TreeTagger (for lemmatization),
megam for learning.

We converted the English side of the text to ascii (XXX: why did we do that?
does Stanford Tagger work better on ASCII? ...) We tokenized both the English
and target language text with the default word tokenizer from NLTK.  We aligned
each of the English and target language pairs (en/de, en/es, etc) with the
Berkeley Aligner, with very nearly the default settings, except that we ran 20
iterations each of IBM Model 1 and 20 iterations of the HMM alignment
algorithm. Is that the right number? Who can say, really?

We also found another bug that made it seem like our alignments were awful.
That was two more problems with TreeTagger -- it turned out that it was calling
both a tokenizer and a mwe-expression chunker. (those could be fixed in the
Perl wrappers for TreeTagger, thankfully)

At this point, we had one-to-many alignments from English words into the target
language sentences.

We also cheated a bit by using the gold standard labels from last time, looking
for instances of the gold labels in the target text; if our noun had an
alignment into those gold standard labels, we preferred taking that gold label
(labels can be multi-word expressions).

We dropped all of the training instances with labels that only occurred once,
considering them likely alignment errors or other noise.

\section{Results}
For the \emph{best} evaluation, the more sophisticated classifiers usually do
better. though not always. It's not totally clear that they're better in
general.

Also, in the out-of-five case, the L1 classifiers are usually better... in
fact, they won the competition for three of the five languages.
However ...

It seems like Els's features really are richer -- she gets translations for all
the different words in the source language and uses the other-target-language
bag-of-words as features. That's a lot of features. We're kind of forcing the
information through a narrower pass -- we just get one decision.

However, in the one-best case, we get better results out of the L2 and MRF
classifiers, so they do seem to help at least a bit. (though the results aren't
all that much better...)

What went wrong with the MRF classifiers with the OOF evaluation??!
TODO(alexr): now we know what went wrong. Write about it!

\begin{table*}[t!]
  \begin{center}
    \begin{tabular}{|r|r|r|r|r|r|}
      \hline
      system   & es    & nl    & de    &  it   & fr \\
      \hline
      baseline & 23.23          & 20.66          & 17.43          & 20.21          & 25.74 \\
   best result & 32.16          & 23.61          & 20.82          & 25.66          & 30.11 \\
            L1 & 29.01          & 21.53          & 19.5           & 24.52          & 27.01 \\
            L2 & 28.49          & \textbf{22.36} & \textbf{19.92} & 23.94          & \textbf{28.23} \\
           MRF & \textbf{29.36} & 21.61          & 19.76          & \textbf{24.62} & 27.46 \\
      \hline
    \end{tabular}
  \caption{``best" evaluation results: precision}
  \label{table:resultsbest}
  \end{center}
\end{table*}


\begin{table*}[t!]
  \begin{center}
    \begin{tabular}{|r|r|r|r|r|r|}
      \hline
      system   & es    & nl    & de    &  it   & fr \\
      \hline
      baseline & 53.07          & 43.59              & 38.86          & 42.63          & 51.36 \\
   best result & 61.69          & 47.83 (proycon c1l)& 44.02          & 53.98          & 59.8 (proycon c1lN) \\
           L1  & 61.69          & 46.55              & 43.66          & 53.57          & 57.76 \\
           L2  & 59.51          & 46.36              & 42.32          & 53.05          & \textbf{58.2} \\
           MRF & \textbf{62.21} & \textbf{46.63}     & \textbf{44.02} & \textbf{53.98} & 57.83 \\
      \hline
    \end{tabular}
  \caption{``oof" evaluation results: precision}
  \label{table:resultsbest}
  \end{center}
\end{table*}


\section{Further experiments}
TODO(alexr): run the experiment where we do the L2 classifiers, but with the
\emph{predicted} target-language word as the feature.

\section{Conclusions and future work}
While our systems had a strong showing in the competition, always beating the
MFS baseline and achieving the top score for three of the five languages in the
OOF evaluation, our system that performed the best during the competition was
the simplest one.

As Lefever and Hoste have shown \cite{lefever-hoste-decock:2011:ACL-HLT2011},
using evidence from multiple parallel corpora can be helpful 

We hope that in the future we will be able to ...
 use more classifiers for the L2 classifier

 the beauty of the L2 classifier approach is that the bitext corpora, they
only have to share a source language -- the sentences in your corpora don't
have to share anything else. Intuitively, it might help for the different
corpora to be different.

We could easily build up more L1 classifiers from any sense-tagged corpus in
our source language. That would be extremely easy.

such as, for example, the sense-tagged corpus distributed with Wordnet:
\footnote{yeah: http://wordnet.princeton.edu/glosstag.shtml}

or the new Google/Amherst wikipedia-tagged corpus, WikiLinks:
\footnote{yeah: http://googleresearch.blogspot.com/2013/03/learning-from-big-data-40-million.html}

We could even use, as L2 features, other monolingual WSD systems. Any
monolingual WSD system, in fact.


an argument against this is that if you have so many bitext corpora, why
don't you just train up a whole SMT system for each language pair? Then you
could do Els's approach of getting bag-of-words features for your
classification. But that's a pain, and do you really want to run Joshua five
times every time you want to translate a noun? IMPORTANT QUESTION: does our L2
work as well as Els on the 2010 trial data? Should run that whole experiment.


come up with better ways to set the edge weights for the MRF. We tried doing
the negative logprob of joint probabilities (and then weighting those with
sort of an arbitrary hyperparameter), and that intuitively seems like it might
be the right thing, but we don't know. We should have tested more. What would
be really great would be a way to estimate these probabilities even if we
can't get them from training data

we could do active learning, actually, to get labels for the same sentence
in multiple target languages. Ask a human "here's a sentence in English, what's
the best translation into target languages X and Y", for the ones we're unsure
about!

or just one target language at a time!

this would be really easy to crowdsource.

this would even be pretty easy to crowdsource for es Guarani, if we have
people who speak Spanish and any other language handy... (gn, en...)

As we're primarily interested in MT for low-resourced and disadvantaged
languages, we plan to adapt this system to Spanish/Guarani MT in the near
future -- earlier experiments were done with a similar, though less
sophisticated, system for Quechua \cite{rudnick:2011:RANLPStud}.

We could also do semi-supervised learning. If we're really confident about
two labels for a given output sentence, then maybe take both of those labels as
training data. (maybe put this in the "possible ridiculous ideas for later"
bin, not in the paper)



\section*{Acknowledgments}

Do not number the acknowledgment section.

\bibliographystyle{naaclhlt2013.bst}
\bibliography{semeval2013.bib}{}

\end{document}
