\documentclass[11pt,letterpaper]{article}
\usepackage{naaclhlt2013}
\usepackage{times}
\usepackage{latexsym}
\setlength\titlebox{6.5cm}    % Expanding the titlebox
\usepackage{float}
\floatstyle{boxed}
\restylefloat{figure}

\title{The HLTDI disambiguatr: some CL-WSD Systems for SemEval Task 10}

\author{Alex Rudnick, Can Liu and Michael Gasser\\
	    Indiana University, School of Informatics and Computing \\
	    {\tt \{alexr,liucan,gasser\}@indiana.edu}}

\date{}

\begin{document}
\maketitle

\begin{abstract}
We present our approaches to CL-WSD for the Semeval 2013 Task 10, which came in
three varieties: "one layer" classifiers, which are single maximum-entropy
classifiers making use of local context features, "two layer" classifiers,
which are the same as the one-layer classifiers except that they use the
translation of the word of interest into four other target languages as
features, and lastly, the "MRF" classifiers, which build a network of five
layer-one classifiers to allow them to solve the classification task jointly.
\end{abstract}

\section{Introduction}
Following the work of Lefever and Hoste
\shortcite{lefever-hoste-decock:2011:ACL-HLT2011}, we are developing systems
that that make use of multiple bitext corpora for the cross-language word-sense
disambiguation (CL-WSD) task, without the need to call a complete MT system at
inference time.

CL-WSD has been shown useful for statistical machine translation, and we are
particularly interested in applying it to rule-based systems.


We presented three systems in this competition, which were variations on the
theme of a maximum entropy classifier for each ambiguous noun, trained on local
context features similar to those used in previous work and familiar from the
WSD literature.


Our systems had similar results, but our simplest system came in first place
for the out-of-five evaluation for three languages. All of our systems beat the
baseline in every case.  In the 2013 competition (XXX cite), winning especially
hard on Flemish, Inuktitut and Oromo (XXX fill this in).


Our three systems made use of the same training data, which we extracted from
the Europarl Intersection corpus, meaning that the English-language source
sentences were identical for each of the five corpora. But the source-language
sentences need not be identical, and we plan to explore relaxing this
restriction in future work.


In the following sections, we will describe our three systems, followed by the
common preprocessing steps that we performed, and then discuss results and
future work.

\section{L1}
preprocessing, training, testing\\
Our simplest system, called "L1" (in contrast with "L2", which we will describe
next), is trained on bitext for a single target language.It makes use of supervised learning to train a maximum entropy (logistic
regression) classifier over a large number of features, described in Figure
\ref{features}. 
We have a ``L1" classifier for each word of interest in each target language.\\  

Most of the work in producing the L1 classifiers is locating relevant training
examples from the bitext corpus. We are only interested in bitex sentence pairs where the source language sentence contains the word of interest.

Once we find translation pairs that we want to include in the training data
(mappings from English-language sentences and instances of the ambiguous word
in question to the appropriate target-language lemma), we extract features from
the English-language sentence. 

Several steps of preprocessing were needed. We first POS tagged the sentences, since we are only interested in nouns.
Then align the words in each sentence pair, and lemmatize the target sentence.
After locating words of interest in the
Europarl Intersection corpus, training instances were extracted, and a maxent
classifier was trained over local context features similar to those used by Lefever and
Hoste.

\begin{figure}
  \begin{itemize}
  \item word form
  \item word tag (word with tag?)
  \item word lemma
  \item for words in the three-word window on both sides
  \item wordform, lemma, pos tag
  \item for words in a five-word window on both sides
  \item bigrams and tagged bigrams (just in case)
  \end{itemize}
  \label{features}
  \caption{some features}
\end{figure}

Note: word tag is different from word with tag (so as for bigram and bigram with tag). eg. ``NN" is a tag for ``bank", but ``bank_NN" is the word with tag. We include this because tags are good indicator of word senses, and the senses of context helps disambiguate the target word. 

\section{L2}
As in the work of Lefever and Hoste, the ``L2" model and MRF model both use translation for four other languages as features, but in different ways.
``L2" model is a maxent classifier with features 1) from English source sentence (the same features as L1), 2) translation for the target word into other four languages. 
At training time, these translations are extracted from the aligned sentences in Europarl Intersection data. 
At testing time, the translations are estimated using the L1 classifiers. 
%Our "L2" model builds on the L1 model, but with the translations of the word of
%interest into the four other target languages as additional features.
The translations seems to be an important factor in the performance of L2, Lefever and Hoste used a MT system for the translations instead.
???

%Unfortunately, it is difficult to evaluate which one gives better performance. It would be a fairer comparison of the WSD systems if a standardized way %of translations 
%So as in the work of Lefever and Hoste, our L2
%classifiers make use of several aligned bitext corpora, but without relying on
%a complete MT system.


\section{MRF}
Our "MRF" model builds a Markov network of L1 classifiers in an effort to find
the best translation into all five target languages jointly. This network has
nodes that correspond to the distribution produced by the L1 classifiers, and
edges with pairwise potentials that are derived from the joint probabilities of
target-language labels occurring together in the training data. We optimize
assignments over the five output variables using loopy belief propagation

The idea of MRF is that translations for five target languages must agree with each other, since these languages are all related.
It is then natural to ask: how related are they and can MRF represent different closeness between languages?
???
We know that Dutch is more related to German than with Spanish, Italian and French. Can this closeness be represented by the pairwise potentials?

We had some concern about pairwise potential in MRF, which is joint probability. Consider a word which occurs 500 times in the training data, it could co-occur with

The future work should experiment on different weight we put on the pairwise potentials...

problem with joint probability as potentials, 
conditional probability as potentials, we could abuse the mathematics of MRF a little bit and experiment on this approximate potential.
\section{resources and tools}
The Europarl Intersection corpus.
	

\section{preprocessing steps}
For simplicity and comparability with previous work, we worked entirely with
the Europarl Intersection corpus provided by the competition organizers.


tools: NLTK, Stanford Tagger, Berkeley Aligner, TreeTagger (for lemmatization),
megam for learning.


We converted the English side of the text to ascii (XXX: why did we do that?
does Stanford Tagger work better on ASCII? ...) We tokenized both the English
and target language text with the default word tokenizer from NLTK.  We aligned
each of the English and target language pairs (en/de, en/es, etc) with the
Berkeley Aligner, with very nearly the default settings, except that we ran 20
iterations each of IBM Model 1 and 20 iterations of the HMM alignment
algorithm. Is that the right number? Who can say, really?


We found an unreasonable bug with TreeTagger that alexr tracked down, which was
breaking our attempts to call TreeTagger programmatically. Helmut fixed the
bug. But he did not release the code to TreeTagger, which is weird.
(TODO(alexr): clone TreeTagger or at least the lemmatizer parts and release an
open source version). Had nobody ever tried to lemmatize English and then
programmatically parse TreeTagger's output before?


We also found another bug that made it seem like our alignments were awful.
That was two more problems with TreeTagger -- it turned out that it was calling
both a tokenizer and a mwe-expression chunker. (those could be fixed in the
Perl wrappers for TreeTagger, thankfully)


Once TreeTagger wasn't screwing up our alignments, our alignments seemed pretty
good, so we used those, such that we had one-to-many alignments from English
words into the target language sentences.


We also cheated a bit by using the gold standard labels from last time, looking
for instances of the gold labels in the target text; if our noun had an
alignment into those gold standard labels, we preferred taking that gold label
(labels can be multi-word expressions).


We dropped all of the training instances with labels that only occurred once,
considering them likely alignment errors or other noise.


\section{results}
So it's kind of surprising -- at least in this case, our simplest system did
the best job. We're comfortable claiming second place for most of the
languages, though Marine did a great job for Spanish.


It's kind of like a negative result -- we were hoping that we could do a better
job with the L2 classifiers or perhaps the MRF classifiers, but really, they
don't improve much on the L1 ones.
So what does that mean?
It seems like Els's features really are richer -- she gets translations for all
the different words in the source language and uses the other-target-language
bag-of-words as features. That's a lot of features. We're kind of forcing the
information through a narrower pass -- we just get one decision.


\section{Conclusions and future work}


While our systems had a strong showing in the competition, always beating the
MFS baseline and achieving the top score for three of the five languages in the
OOF evaluation, our system that performed the best was the simplest one.


We hope that in the future we will be able to ...
 use more classifiers for the L2 classifier

 the beauty of the L2 classifier approach is that the bitext corpora, they
only have to share a source language -- the sentences in your corpora don't
have to share anything else. Intuitively, it might help for the different
corpora to be different.

 we could easily build up more L1 classifiers from any sense-tagged corpus in
our source language. That would be extremely easy.

such as, for example, the sense-tagged corpus distributed with Wordnet:
http://wordnet.princeton.edu/glosstag.shtml


or the new Google/Amherst wikipedia-tagged corpus, WikiLinks:
http://googleresearch.blogspot.com/2013/03/learning-from-big-data-40-million.html


we could even use, as L2 features, other monolingual WSD systems. Any
monolingual WSD system, in fact.


an argument against this is that if you have so many bitext corpora, why
don't you just train up a whole SMT system for each language pair? Then you
could do Els's approach of getting bag-of-words features for your
classification. But that's a pain, and do you really want to run Joshua five
times every time you want to translate a noun? IMPORTANT QUESTION: does our L2
work as well as Els on the 2010 trial data? Should run that whole experiment.


come up with better ways to set the edge weights for the MRF. We tried doing
the negative logprob of joint probabilities (and then weighting those with
sort of an arbitrary hyperparameter), and that intuitively seems like it might
be the right thing, but we don't know. We should have tested more. What would
be really great would be a way to estimate these probabilities even if we
can't get them from training data

we could do active learning, actually, to get labels for the same sentence
in multiple target languages. Ask a human "here's a sentence in English, what's
the best translation into target languages X and Y", for the ones we're unsure
about!

or just one target language at a time!

this would be really easy to crowdsource.

this would even be pretty easy to crowdsource for es Guarani, if we have
people who speak Spanish and any other language handy... (gn, en...)

We could also do semi-supervised learning. If we're really confident about
two labels for a given output sentence, then maybe take both of those labels as
training data. (maybe put this in the "possible ridiculous ideas for later"
bin, not in the paper)


Els 2013: the shared task paper.

(Maybe cite Fran's dissertation on how he's doing lexical selection for
Apertium)

\cite{task10}

\section*{Acknowledgments}

Do not number the acknowledgment section.

\bibliographystyle{naaclhlt2013.bst}
\bibliography{semeval2013.bib}{}

\end{document}
